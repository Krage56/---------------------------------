\documentclass{article}

\usepackage{tikz}%графы  

\usepackage{cmap}                   
\usepackage{mathtext}   
\usepackage{mathtools}
\usepackage{amssymb,amsmath}
\usepackage{amsthm}
\usepackage{tipx}
\usepackage{tabularx,array,booktabs}
\usepackage{algorithm2e}
\usepackage{listings}
\usepackage{hyperref}
%изображения
\usepackage{graphicx}%Вставка картинок правильная
\usepackage{float}%"Плавающие" картинки
\usepackage{wrapfig}%Обтекание фигур (таблиц, картинок и прочего)
\usepackage{setspace}
\usepackage[normalem]{ulem}
\usepackage{arcs}
\usepackage[T1,T2A]{fontenc}        
\usepackage[utf8]{inputenc}         
\usepackage[english, russian]{babel} 
\usepackage[top=0.35in, bottom=0.5in, left=0.3in,right=0.3in]{geometry}
\usepackage{scalerel}
\usepackage{stackengine}
\stackMath
\newtheorem{theorem}{Теорема}
\newtheorem{definition}{Определение}
\newtheorem{proposal}{Предположение}
\newtheorem{notice}{Замечание}
\newtheorem{statement}{Положение}
\newtheorem{corollary}{Следствие}
\newtheorem{lemma}{Лемма}
\newtheorem{observation}{Наблюдение}
\newtheorem{problem}{Задача}
\newtheorem{claim}{Решение}
\newtheorem{explanation}{Пояснение}
\newtheorem*{explanation*}{Пояснение}

\newcommand{\RomanNumeralCaps}[1]
    {\MakeUppercase{\romannumeral #1}}
\newenvironment{shortlist}
  {\renewcommand{\item}{\renewcommand{\item}{\unskip\space\textbullet~}}}
  {}
\renewcommand{\tabularxcolumn}[1]{m{#1}}    
\newcommand\reallywidesmile[1]{%
\stackon[0.5pt]{#1}{%
\stretchto{%
  \scaleto{%
    \scalerel*[\widthof{#1}]{\mkern-1.5mu\smile\mkern-2mu}%
    {\rule[-\textheight/2]{1ex}{\textheight}}%
  }{\textheight}%
}{0.8ex}}%
}
\parskip 1ex


\begin{document}
\section*{Семинары по дискретной математике}
\section{Автоморфизмы графа. Планарные графы. Поток на графе}
\subsection{Автоморфизм графа}
\begin{definition}
    \textbf{Граф} - это множенство точек $V$ и множенство ориентированных
     рёбер $\overrightarrow{E}$,
    на которых заданы отображения $S:\overrightarrow{E} \rightarrow V$ и $t:\overrightarrow{E} \rightarrow V$ такие,
    что $t(e)=v \Rightarrow \phi(t(e)) = \phi(v)$.
\end{definition}
\begin{explanation*}
    Вот, что такое отображения выше:
    \begin{itemize}
        \item $S:\overrightarrow{E} \rightarrow V$ - функция, возвращающая
        начало ребра
        \item$t:\overrightarrow{E} \rightarrow V$ - функция, возвращающая конец ребра
        \item Последние условие на отображения говорит нам о сохранении структуры графа
    \end{itemize}
\end{explanation*}

\begin{definition}
    \textbf{Изоморфизм графов} - это биективное отображение графа $G$ на граф $H$ с сохранением
    структуры.
\end{definition}
\begin{explanation*}
    Пусть $G, H$ - графы, тогда биективные отображения
        $$\phi : V_{G} \rightarrow V_{H}$$
        $$\psi : \overrightarrow{E}_{G} \rightarrow \overrightarrow{E}_{H}$$
задающие биекции, с сохранением структуры, задают изоморфизм.
\end{explanation*}
\textbf{Пример:} даны графы $G, H$ соответственно

\begin{tikzpicture}
    \begin{scope}[every node/.style={circle,thick,draw}]
        \node[shape=circle,draw=black] (A) at (0,1.5) {$a$};
        \node[shape=circle,draw=black] (B) at (-1.5,0) {$b$};
        \node[shape=circle,draw=black] (C) at (-1.25,-2) {$c$};
        \node[shape=circle,draw=black] (D) at (1.25,-2) {$d$};
        \node[shape=circle,draw=black] (E) at (1.5,0) {$e$};
    \end{scope}
    
    \begin{scope}[every node/.style={circle,thick,draw}]
    \path (A) edge (B);
    \path (A) edge (E);
    \path (B) edge (C);
    \path (E) edge (D);
    \path (D) edge (C);
    \end{scope}

    \begin{scope}[every node/.style={circle,thick,draw}]
        \node[shape=circle,draw=black] (A) at (6,1.5) {$\alpha$};
        \node[shape=circle,draw=black] (B) at (6-1.5,0) {$\varepsilon$};
        \node[shape=circle,draw=black] (C) at (6-1.25,-2) {$\delta$};
        \node[shape=circle,draw=black] (D) at (6+1.25,-2) {$\gamma$};
        \node[shape=circle,draw=black] (E) at (6+1.5,0) {$\beta$};
    \end{scope}
    \begin{scope}[every node/.style={circle,thick,draw}]
        \path (A) edge (D);
        \path (A) edge (C);
        \path (B) edge (C);
        \path (B) edge (A);
        \path (E) edge (D);
        \path (E) edge (A);
        \path (E) edge (B);
        \path (E) edge (C);
        \path (B) edge (D);
        \path (D) edge (C);
    \end{scope}

    \node[text width=6cm, anchor=west, right] at (7.5,0)
    {
        \begin{multline*}
            \phi : V_{G} \rightarrow V_{H}\text{ - биективное отображение, тогда:}\\
            \phi(a) = \alpha,\\
            \phi(b) = \gamma, \\
            \phi(c) = \varepsilon, \\
            \phi(d) = \beta, \\
            \phi(e) = \delta\\
        \end{multline*}
    };
    \end{tikzpicture}
    
\end{document}